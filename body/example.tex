\chapter*{藤野先生}\writer{鲁迅} % \chapter*{标题}\writer{作者}
\addcontentsline{toc}{chapter}{藤野先生} % \addcontentsline{toc}{chapter}{标题} (此举是为了在目录上显示标题)

    东京也无非是这样。上野的樱花烂熳的时节,望去确也像绯红的轻云,但花下也缺不了成群结队的“清国留学生”的速成班,头顶上盘着大辫子,顶得学生制帽的顶上高高耸起,形成一座富士山。也有解散辫子,盘得平的,除下帽来,油光可鉴,宛如小姑娘的发髻一般,还要将脖子扭几扭。实在标致极了。

    中国留学生会馆的门房里有几本书买,有时还值得去一转;倘在上午,里面的几间洋房里倒也还可以坐坐的。但到傍晚,有一间的地板便常不免要咚咚咚地响得震天,兼以满房烟尘斗乱;问问精通时事的人,答道,“那是在学跳舞。”

    到别的地方去看看,如何呢?

    我就往仙台的医学专门学校去。从东京出发,不久便到一处驿站,写道: 日暮里。不知怎地,我到现在还记得这名目。其次却只记得水户了,这是明的遗民朱舜水先生客死的地方。仙台是一个市镇,并不大;冬天冷得利害; 还没有中国的学生。

    大概是物以希为贵罢。北京的白菜运往浙江,便用红头绳系住菜根,倒挂在水果店头,尊为“胶菜”; 福建野生着的芦荟,一到北京就请进温室,且美其名曰“龙舌兰”。我到仙台也颇受了这样的优待,不但学校不收学费,几个职员还为我的食宿操心。我先是住在监狱旁边一个客店里的,初冬已经颇冷,蚊子却还多,后来用被盖了全身,用衣服包了头脸,只留两个鼻孔出气。在这呼吸不息的地方,蚊子竟无从插嘴,居然睡安稳了。饭食也不坏。但一位先生却以为这客店也包办囚人的饭食,我住在那里不相宜,几次三番,几次三番地说。我虽然觉得客店兼办囚人的饭食和我不相干,然而好意难却,也只得别寻相宜的住处了。于是搬到别一家,离监狱也很远,可惜每天总要喝难以下咽的芋梗汤。

    从此就看见许多陌生的先生,听到许多新鲜的讲义。解剖学是两个教授分任的。最初是骨学。其时进来的是一个黑瘦的先生,八字须,戴着眼镜,挟着一叠大大小小的书。一将书放在讲台上,便用了缓慢而很有顿挫的声调,向学生介绍自己道:

    “我就是叫作藤野严九郎的……”

    后面有几个人笑起来了。他接着便讲述解剖学在日本发达的历史,那些大大小小的书,便是从最初到现今关于这一门学问的著作。起初有几本是线装的;还有翻刻中国译本的,他们的翻译和研究新的医学,并不比中国早。

    那坐在后面发笑的是上学年不及格的留级学生,在校已经一年,掌故颇为熟悉的了。他们便给新生讲演每个教授的历史。这藤野先生,据说是穿衣服太模胡了,有时竟会忘记带领结;冬天是一件旧外套,寒颤颤的,有一回上火车去,致使管车的人疑心他是扒手,叫车里的客人大家小心些。

    他们的话大概是真的,我就亲见他有一次上讲堂没有带领结。

    过了一星期,大约是星期六,他使助手来叫我了。到得研究室,见他坐在人骨和许多单独的头骨中间,——他其时正在研究着头骨,后来有一篇论文在本校的杂志上发表出来。

    “我的讲义,你能抄下来么?”他问。

    “可以抄一点。”

    “拿来我看!”

    我交出所抄的讲义去,他收下了,第二三天便还我,并且说,此后每一星期要送给他看一回。我拿下来打开看时,很吃一惊,同时也感到一种不安和感激。原来我的讲义已经从头到末,都用红笔添改过了,不但增加了许多脱漏的地方,连文法的错误,也都一一订正。这样一直继续到教完了他所担任的功课: 骨学,血管学,神经学。

    可惜我那时太不用功,有时也很任性。还记得有一回藤野先生将我叫到他的研究室里去,翻出我那讲义上的一个图来,是下臂的血管,指着,向我和蔼的说道:

    “你看,你将这条血管移了一点位置了。——自然,这样一移,的确比较的好看些,然而解剖图不是美术,实物是那么样的,我们没法改换它。现在我给你改好了,以后你要全照着黑板上那样的画。”

    但是我还不服气,口头答应着,心里却想道:

    “图还是我画的不错;至于实在的情形,我心里自然记得的。”

    学年试验完毕之后,我便到东京玩了一夏天,秋初再回学校,成绩早已发表了,同学一百余人之中,我在中间,不过是没有落第。这回藤野先生所担任的功课,是解剖实习和局部解剖学。

    解剖实习了大概一星期,他又叫我去了,很高兴地,仍用了极有抑扬的声调对我说道:

    “我因为听说中国人是很敬重鬼的,所以很担心,怕你不肯解剖尸体。现在总算放心了,没有这回事。”

    但他也偶有使我很为难的时候。他听说中国的女人是裹脚的,但不知道详细,所以要问我怎么裹法,足骨变成怎样的畸形,还叹息道,“总要看一看才知道。究竟是怎么一回事呢?”

    有一天,本级的学生会干事到我寓里来了,要借我的讲义看。我检出来交给他们,却只翻检了一通,并没有带走。但他们一走,邮差就送到一封很厚的信,拆开看时,第一句是:

    “你改悔罢!”

    这是《新约》上的句子罢,但经托尔斯泰新近引用过的。其时正值日俄战争,托老先生便写了一封给俄国和日本的皇帝的信,开首便是这一句。日本报纸上很斥责他的不逊,爱国青年也愤然,然而暗地里却早受了他的影响了。其次的话,大略是说上年解剖学试验的题目,是藤野先生在讲义上做了记号,我预先知道的,所以能有这样的成绩。末尾是匿名。

    我这才回忆到前几天的一件事。因为要开同级会,干事便在黑板上写广告,末一句是“请全数到会勿漏为要”,而且在“漏”字旁边加了一个圈。我当时虽然觉到圈得可笑,但是毫不介意,这回才悟出那字也在讥刺我了,犹言我得了教员漏泄出来的题目。

    我便将这事告知了藤野先生; 有几个和我熟识的同学也很不平,一同去诘责干事托辞检查的无礼,并且要求他们将检查的结果,发表出来。终于这流言消灭了,干事却又竭力运动,要收回那一封匿名信去。结末是我便将这托尔斯泰式的信退还了他们。

    中国是弱国,所以中国人当然是低能儿,分数在六十分以上,便不是自己的能力了:也无怪他们疑惑。但我接着便有参观枪毙中国人的命运了。第二年添教霉菌学,细菌的形状是全用电影来显示的,一段落已完而还没有到下课的时候,便影几片时事的片子,自然都是日本战胜俄国的情形。但偏有中国人夹在里边:给俄国人做侦探,被日本军捕获,要枪毙了,围着看的也是一群中国人; 在讲堂里的还有一个我。

    “万岁!”他们都拍掌欢呼起来。

    这种欢呼,是每看一片都有的,但在我,这一声却特别听得刺耳。此后回到中国来,我看见那些闲看枪毙犯人的人们,他们也何尝不酒醉似的喝采,——呜呼,无法可想!但在那时那地,我的意见却变化了。

    到第二学年的终结,我便去寻藤野先生,告诉他我将不学医学,并且离开这仙台。他的脸色仿佛有些悲哀,似乎想说话,但竟没有说。

    “我想去学生物学,先生教给我的学问,也还有用的。”其实我并没有决意要学生物学,因为看得他有些凄然,便说了一个慰安他的谎话。

    “为医学而教的解剖学之类,怕于生物学也没有什么大帮助。” 他叹息说。

    将走的前几天,他叫我到他家里去,交给我一张照相,后面写着两个字道:“惜别”,还说希望将我的也送他。但我这时适值没有照相了;他便叮嘱我将来照了寄给他,并且时时通信告诉他此后的状况。

    我离开仙台之后,就多年没有照过相,又因为状况也无聊,说起来无非使他失望,便连信也怕敢写了。经过的年月一多,话更无从说起,所以虽然有时想写信,却又难以下笔,这样的一直到现在,竟没有寄过一封信和一张照片。从他那一面看起来,是一去之后,杳无消息了。

    但不知怎地,我总还时时记起他,在我所认为我师的之中,他是最使我感激,给我鼓励的一个。有时我常常想:他的对于我的热心的希望,不倦的教诲,小而言之,是为中国,就是希望中国有新的医学;大而言之,是为学术,就是希望新的医学传到中国去。他的性格,在我的眼里和心里是伟大的,虽然他的姓名并不为许多人所知道。

    他所改正的讲义,我曾经订成三厚本,收藏着的,将作为永久的纪念。不幸七年前迁居的时候,中途毁坏了一口书箱,失去半箱书,恰巧这讲义也遗失在内了。责成运送局去找寻,寂无回信。只有他的照相至今还挂在我北京寓居的东墙上,书桌对面。每当夜间疲倦,正想偷懒时,仰面在灯光中瞥见他黑瘦的面貌,似乎正要说出抑扬顿挫的话来,便使我忽又良心发现,而且增加勇气了,于是点上一枝烟,再继续写些为 “正人君子”之流所深恶痛疾的文字。

    \hfill 十月十二日

