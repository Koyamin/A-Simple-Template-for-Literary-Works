\articleinfo{拿来主义}{鲁迅}

    中国一向是所谓“闭关主义”,自己不去,别人也不许来。自从给枪炮打破了大门之后,又碰了一串钉子,到现在,成了什么都是“送去主义”了。别的且不说罢,单是学艺上的东西,近来就先送一批古董到巴黎去展览,但终“不知后事如何”;还有几位“大师”们捧着几张古画和新画,在欧洲各国一路的挂过去,叫作“发扬国光”。听说不远还要送梅兰芳博士到苏联去,以催进“象征主义”,此后是顺便到欧洲传道。我在这里不想讨论梅博士演艺和象征主义的关系,总之,活人替代了古董,我敢说,也可以算得显出一点进步了。

    但我们没有人根据了“礼尚往来”的仪节,说道:拿来!

    当然,能够只是送出去,也不算坏事情,一者见得丰富,二者见得大度。尼采就自诩过他是太阳,光热无穷,只是给与,不想取得。然而尼采究竟不是太阳,他发了疯。中国也不是,虽然有人说,掘起地下的煤来,就足够全世界几百年之用,但是,几百年之后呢?几百年之后,我们当然是化为魂灵,或上天堂,或落了地狱,但我们的子孙是在的,所以还应该给他们留下一点礼品。要不然,则当佳节大典之际,他们拿不出东西来,只好磕头贺喜,讨一点残羹冷炙做奖赏。

    这种奖赏,不要误解为“抛来”的东西,这是“抛给”的,说得冠冕些,可以称之为“送来”,我在这里不想举出实例。

    我在这里也并不想对于“送去”再说什么,否则太不“摩登”了。我只想鼓吹我们再吝啬一点,“送去”之外,还得“拿来”,是为“拿来主义”。

    但我们被“送来”的东西吓怕了。先有英国的鸦片,德国的废枪炮,后有法国的香粉,美国的电影,日本的印着“完全国货”的各种小东西。于是连清醒的青年们,也对于洋货发生了恐怖。其实,这正是因为那是“送来”的,而不是“拿来”的缘故。

    所以我们要运用脑髓,放出眼光,自己来拿!

    譬如罢,我们之中的一个穷青年,因为祖上的阴功(姑且让我这么说说罢),得了一所大宅子,且不问他是骗来的,抢来的,或合法继承的,或是做了女婿换来的。那么,怎么办呢?我想,首先是不管三七二十一,“拿来”!但是,如果反对这宅子的旧主人,怕给他的东西染污了,徘徊不敢走进门,是孱头;勃然大怒,放一把火烧光,算是保存自己的清白,则是昏蛋。不过因为原是羡慕这宅子的旧主人的,而这回接受一切,欣欣然的蹩进卧室,大吸剩下的鸦片,那当然更是废物。“拿来主义”者是全不这样的。

    他占有,挑选。看见鱼翅,并不就抛在路上以显其“平民化”,只要有养料,也和朋友们像萝卜白菜一样的吃掉,只不用它来宴大宾;看见鸦片,也不当众摔在茅厕里,以见其彻底革命,只送到药房里去,以供治病之用,却不弄“出售存膏,售完即止”的玄虚。只有烟枪和烟灯,虽然形式和印度,波斯,阿剌伯的烟具都不同,确可以算是一种国粹,倘使背着周游世界,一定会有人看,但我想,除了送一点进博物馆之外,其余的是大可以毁掉的了。还有一群姨太太,也大以请她们各自走散为是,要不然,“拿来主义”怕未免有些危机。

    总之,我们要拿来。我们要或使用,或存放,或毁灭。那么,主人是新主人,宅子也就会成为新宅子。然而首先要这人沉着,勇猛,有辨别,不自私。没有拿来的,人不能自成为新人,没有拿来的,文艺不能自成为新文艺。

    六月四日。