\articleinfo{中国语文的新生}{鲁迅}

    中国现在的所谓中国字和中国文,已经不是中国大家的东西了。

    古时候,无论那一国,能用文字的原是只有少数的人的,但到现在,教育普及起来,凡是称为文明国者,文字已为大家所公有。但我们中国,识字的却大概只占全人口的十分之二,能作文的当然还要少。这还能说文字和我们大家有关系么?

    也许有人要说,这十分之二的特别国民,是怀抱着中国文化,代表着中国大众的。我觉得这话并不对。这样的少数,并不足以代表中国人。正如中国人中,有吃燕窝鱼翅的人,有卖红丸的人,有拿回扣的人,但不能因此就说一切中国人,都在吃燕窝鱼翅,卖红丸,拿回扣一样。要不然,一个郑孝胥〔2〕,真可以把全副“王道”挑到满洲去。

    我们倒应该以最大多数为根据,说中国现在等于并没有文字。

    这样的一个连文字也没有的国度,是在一天一天的坏下去了。我想,这可以无须我举例。

    单在没有文字这一点上,智识者是早就感到模胡的不安的。清末的办白话报,五四时候的叫“文学革命”,就为此。但还只知道了文章难,没有悟出中国等于并没有文字。今年的提倡复兴文言文,也为此,他明知道现在的机关枪是利器,却因历来偷懒,未曾振作,临危又想侥幸,就只好梦想大刀队成事了。

    大刀队的失败已经显然,只有两年,已没有谁来打九十九把钢刀去送给军队〔3〕。但文言队的显出不中用来,是很慢,很隐的,它还有寿命。

    和提倡文言文的开倒车相反,是目前的大众语文的提倡,但也还没有碰到根本的问题:中国等于并没有文字。待到拉丁化的提议出现,这才抓住了解决问题的紧要关键。

    反对,当然大大的要有的,特殊人物的成规,动他不得。格理莱〔4〕倡地动说,达尔文〔5〕说进化论,摇动了宗教,道德的基础,被攻击原是毫不足怪的;但哈飞〔6〕发见了血液在人身中环流,这和一切社会制度有什么关系呢,却也被攻击了一世。然而结果怎样?结果是:血液在人身中环流!

    中国人要在这世界上生存,那些识得《十三经》的名目的学者,“灯红”会对“酒绿”的文人,并无用处,却全靠大家的切实的智力,是明明白白的。那么,倘要生存,首先就必须除去阻碍传布智力的结核:非语文和方块字。如果不想大家来给旧文字做牺牲,就得牺牲掉旧文字。走那一面呢,这并非如冷笑家所指摘,只是拉丁化提倡者的成败,乃是关于中国大众的存亡的。要得实证,我看也不必等候怎么久。

    至于拉丁化的较详的意见,我是大体和《自由谈》连载的华圉作《门外文谈》相近的,这里不多说。我也同意于一切冷笑家所冷嘲的大众语的前途的艰难;但以为即使艰难,也还要做;愈艰难,就愈要做。改革,是向来没有一帆风顺的,冷笑家的赞成,是在见了成效之后,如果不信,可看提倡白话文的当时。

    \hfill 九月二十四日