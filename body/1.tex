\articleinfo{中国人失掉自信力了吗}{周树人}

    从公开的文字上看起来:两年以前,我们总自夸着“地大物博”,是事实;不久就不再自夸了,只希望着国联,也是事实;现在是既不夸自己,也不信国联,改为一味求神拜佛,怀古伤今了——却也是事实。

    于是有人慨叹曰: 中国人失掉自信力了。

    如果单据这一点现象而论,自信其实是早就失掉了的。先前信“地”,信“物”,后来信“国联”,都没有相信过“自己”。假使这也算一种“信”,那也只能说中国人曾经有过“他信力”,自从对国联失望之后,便把这他信力都失掉了。

    失掉了他信力,就会疑,一个转身,也许能够只相信了自己,倒是一条新生路,但不幸的是逐渐玄虚起来了。信“地”和“物”,还是切实的东西,国联就渺茫,不过这还可以令人不久就省悟到依赖它的不可靠。一到求神拜佛,可就玄虚之至了,有益或是有害,一时就找不出分明的结果来,它可以令人更长久的麻醉着自己。

    中国人现在是在发展着 “自欺力”。

    “自欺”也并非现在的新东西,现在只不过日见其明显,笼罩了一切罢了。然而,在这笼罩之下,我们有并不失掉自信力的中国人在。

    我们从古以来,就有埋头苦干的人,有拚命硬干的人,有为民请命的人,有舍身求法的人,……虽是等于为帝王将相作家谱的所谓“正史”,也往往掩不住他们的光耀,这就是中国的脊梁。

    这一类的人们,就是现在也何尝少呢? 他们有确信,不自欺;他们在前仆后继的战斗,不过一面总在被摧残,被抹杀,消灭于黑暗中,不能为大家所知道罢了。说中国人失掉了自信力,用以指一部分人则可,倘若加于全体,那简直是诬蔑。

    要论中国人,必须不被搽在表面的自欺欺人的脂粉所诓骗,却看看他的筋骨和脊梁。自信力的有无,状元宰相的文章是不足为据的,要自己去看地底下。

    \hfill 九月二十五日

